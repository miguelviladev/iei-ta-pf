\documentclass{report}
\usepackage[T1]{fontenc}
\usepackage[utf8]{inputenc}
\usepackage[backend=biber, style=ieee]{biblatex}
\usepackage{csquotes}
\usepackage[portuguese]{babel}
\usepackage{blindtext}
\usepackage[printonlyused]{acronym}
\usepackage{hyperref}
\usepackage{graphicx}
\usepackage{indentfirst}
\bibliography{bibliografia}

\begin{document}
%%% DEFINIÇÕES GLOBAIS %%%
\def\titulo{Realidade Aumentada}
\def\data{Aveiro, dezembro 2021}
\def\autores{Miguel Vila, Diogo Silva}
\def\autorescontactos{(107276) miguelovila@ua.pt, (108212) dsgps@ua.pt}
\def\versao{BETA SINCE 2013}
\def\departamento{DETI}
\def\empresa{UNIVERSIDADE DE AVEIRO}
\def\logotipo{ua.pdf}

%%% ESTRUTURA CAPA %%%
\begin{titlepage}
\begin{center}
\vspace*{50mm}
{\Huge \titulo}\\ 
\vspace{10mm}
{\Large \empresa}\\
\vspace{10mm}
{\LARGE \autores}\\ 
\vspace{30mm}
\begin{figure}[h]
\center
\includegraphics{\logotipo}
\end{figure}
\vspace{30mm}
\end{center}
\begin{flushright}
\versao
\end{flushright}
\end{titlepage}

%%%  PÁGINA DE TÍTULO %%%
\title{%
{\Huge\textbf{\titulo}}\\
{\Large \departamento\\ \empresa}
}
\author{
    \autores \\
    \autorescontactos
}
\date{\data}
\maketitle
\pagenumbering{roman}

%%% RESUMO %%%
\begin{abstract}
!!!TODO!!! Resumo de 200-300 palavras.
\end{abstract}

%%% AGRADECIMENTOS %%%
\renewcommand{\abstractname}{Agradecimentos}
\begin{abstract}
!!!TODO!!!
\end{abstract}

\tableofcontents
% \listoftables     % descomentar se necessário
% \listoffigures    % descomentar se necessário

\clearpage
\pagenumbering{arabic}

%%% INTRODUÇÃO %%%%
\chapter{Introdução}
\label{chap.introducao}
Desde os primórdios que o Homem procura ter controlo da sua realidade moldando-a e modificando-a de modo a que as suas necessidades sejam supridas. Pode-se tomar como exemplo o controlo do fogo: quando o Homem primitivo descobriu como gerar artificialmente e controlar o fogo teve a sua vida facilitada e abriu um leque de novas possibilidades que originaram uma grande revolução a todos os níveis.

Passados alguns milhares de anos, o ser humano continua a tentar ter ainda mais controlo sobre a realidade de modo a que o impossível se torne possível. Como a \ac{ra} extende virtualmente aquilo que existe no mundo real, existe uma forte probabilidade de que, tal como o fogo, a \ac{ra} venha a revolucionar a forma como se vive e dar azo ao surgimento de novas possibilidades.

Apesar de ser uma tecnologia relativamente recente, esta tem tido uma considerável evolução por isso, promete ser o futuro da tecnologia e integrar-se cada vez mais no dia a dia do cidadão comum. Atualmente não está implementada em grande escala, mas já tem vastas aplicações a nível empresarial. Áreas como a medicina, o entretenimento, o \textit{design}, a educação e a arquitetura poderão beneficiar dos novos recursos e funcionalidades criados por esta tecnologia.

Além disso, empresas no mercado tecnológico como a Google, a Microsoft e a Samsung apostam no desenvolvimento desta tecnologia que tem potencial para se tornar o “braço-direito” do utilizador no desenvolver da sua atividade profissional e, futuramente, no desenvolvimento da sua vida pessoal. Porém, atualmente apenas a Microsoft foi capaz de, com algum sucesso, viabilizar e introduzir estes dispositivos no ambiente industrial e corporativo.

\section{Objetivos}
Este relatório, realizado no âmbito da unidade curricular de Introdução à Engenharia Informática, terá como principal objetivo dar a conhecer a nova realidade tecnológica dos dispositivos \textit{mixed reality} e a sua utilidade, focando nos óculos holográficos de realidade aumentada HoloLens desenvolvidos pela Microsoft. 

\section{Organização e estrutura}
O relatório \textit{Realidade Aumentada} contém informação relativa a esta área e está organizado de forma a que o leitor não necessite de quaisquer conhecimentos prévios para poder acompanhar a totalidade dos capítulos.

\section{Metodologia}
Na elaboração deste relatório utilizou-se uma metodologia assente na pesquisa exploratória. Esta pesquisa qualitativa baseou-se em diferentes fontes tais como revistas de referência na área da tecnologia e ciência e investigações nestas mesmas áreas e permitiu entender e aprofundar o tema escolhido para que este fosse abordado da maneira mais clara e completa possível.

%%% REALIDADE AUMENTADA %%%
\chapter{Realidade Aumentada}
\label{chap.a-realidade-aumentada}
!!!TODO!!!

\section{Conceito}
Realidade Aumentada \ac{ra} ou \ac{ar} consiste na integração de elementos ou informações virtuais na visualização do mundo real através de uma câmera, com o uso de sensores de movimento como o giroscópio e o acelerômetro. O uso mais utilizado, e mais conhecido da realidade aumentada é o entretenimento, através dos filtros para fotos em aplicativos móveis de redes sociais, através de jogos como o Pokémon GO. A realidade aumentada é também utilizada de muitas formas nas áreas do ensino, design de produtos, ações de marketing, suporte em plantas industriais, entre outros. O uso de vídeos transmitidos ao vivo digitalmente processados e "ampliados" pela adição de gráficos criados pelo computador também podem ser considerados como um tipo de realidade aumentada. Um usuário da \ac{ra} pode utilizar uns óculos, ou câmeras acopladas a um dispositivo computacional, e através destes, poderá ver o mundo real bem como imagens geradas por computador projetadas no mundo.

A \ac{ar} baseia-se numa experiência interativa entre um mundo real, onde objetos que pertencem ao mundo real podem ser alterados por informação perceptiva criada por computadores, podendo ser visual, auditiva, sensorial e olfativo. Pode ser construtiva (que acrescenta ao ambiente natural) ou destrutiva (que mascarpõe uma mascara sobreposta ao ambiente natural). A realidade aumentada altera o mundo real do usuário, enquanto a \ac{rv} substitui completamente o mundo real do expectador. A Realidade aumentada é relacionada a dois termos muito usados no meio tecnologico: a Realidade mista, e a Realidade mediada por computadores. 

-Realidade Mista - a realidade mista é a tecnologia que une as características da realidade virtual com a realidade aumentada. Permite inserir objetos virtuais num mundo real e permite a interação do usuário com os mesmos, produzindo um novo ambiente ao qual os itens físicos e virtuais coexistem e interagem em tempo real. Um exemplo da realidade mista é o \textit{head-up display} que encontramos nos carros mais modernos.

-Realidade Mediada - a realidade mediada consiste na capacidade de adicionar ou subtrair informação da precepção da realidade através da Utilização de um \textit{wearable computer} ou mesmo de um \textit{smartphone}, basta um dispositivo que permita criar um filtro visual entre o mundo real e aquilo que o utilizador capta, criando um cenário novo ao utiliador.

Voltando a falar da \ac{ra}, é ela que permite trazer componentes do mundo digital para dentro da percepção da pessoa do mundo real, e não o faz apenas dispondo as informações visualmente, mas também através da integração de sensações imersivas que são interpretadas como sendo algo pertencente a um ambiente.

\section{Origem}
!!!TODO!!!

\section{Aplicações}
!!!TODO!!!

\subsection{Exemplos}
!!!TODO!!!

%%% ÓCULOS HOLOGRÁFICOS %%%
\chapter{Óculos Holográficos}
\label{chap.oculos-holograficos}
!!!TODO!!!

\section{Conceito}
!!!TODO!!!

\section{Panorama atual}
!!!TODO!!!

%%% MICROSOFT HOLOLENS 2 %%%
\chapter{Microsoft HoloLens 2}
\label{chap.microsoft-hololens-2}
!!!TODO!!!

\section{Proposta do produto}
!!!TODO!!!

\section{Design e principais características}
!!!TODO!!!

\section{Público Alvo}
!!!TODO!!!

%%% CONCLUSÕES %%%
\chapter{Conclusões}
\label{chap.conclusao}
!!!TODO!!!

\chapter*{Contribuições dos autores}
!!!TODO!!! Ambos paricipamos ativamente e com empenho, procurando contribuir para a realização dum trabalho com boa apresentação e conteúdo.

Resumir aqui o que cada autor fez no trabalho. Usar abreviaturas para identificar os autores, por exemplo AS para António Silva. No fim indicar a percentagem de contribuição de cada autor.

%%% ACRÓNIMOS %%%
\chapter*{Acrónimos}
\begin{acronym}
\acro{ua}[UA]{Universidade de Aveiro}
\acro{leci}[LECI]{Licenciatura em Engenharia de Computadores e Informática}
\acro{ra}[RA]{Realidade Aumentada}
\acro{ar}[AR]{Augmented Reality}
\acro{rv}[RV]{Realidade Virtual}
\end{acronym}

%%% BIBLIOGRAFIA %%%
\printbibliography

\end{document}