\documentclass{report}
\usepackage[T1]{fontenc} % Fontes T1
\usepackage[utf8]{inputenc} % Input UTF8
\usepackage[backend=biber, style=ieee]{biblatex} % para usar bibliografia
\usepackage{csquotes}
\usepackage[portuguese]{babel} %Usar língua portuguesa
\usepackage{blindtext} % Gerar texto automaticamente
\usepackage[printonlyused]{acronym}
\usepackage{hyperref} % para autoref
\usepackage{graphicx}
\usepackage{indentfirst}

\bibliography{bibliografia}


\begin{document}
%%
% Definições
%
\def\titulo{Realidade Aumentada}
\def\data{Aveiro, dezembro 2021}
\def\autores{Miguel Vila, Diogo Silva}
\def\autorescontactos{(107276) miguelovila@ua.pt, (108212) dsgps@ua.pt}
\def\versao{BETA SINCE 2013}
\def\departamento{DETI}
\def\empresa{UNIVERSIDADE DE AVEIRO}
\def\logotipo{ua.pdf}
%
%%%%%% CAPA %%%%%%
%
\begin{titlepage}

\begin{center}
%
\vspace*{50mm}
%
{\Huge \titulo}\\ 
%
\vspace{10mm}
%
{\Large \empresa}\\
%
\vspace{10mm}
%
{\LARGE \autores}\\ 
%
\vspace{30mm}
%
\begin{figure}[h]
\center
\includegraphics{\logotipo}
\end{figure}
%
\vspace{30mm}
\end{center}
%
\begin{flushright}
\versao
\end{flushright}
\end{titlepage}

%%  Página de Título %%
\title{%
{\Huge\textbf{\titulo}}\\
{\Large \departamento\\ \empresa}
}
%
\author{%
    \autores \\
    \autorescontactos
}
%
\date{\data}
%
\maketitle

\pagenumbering{roman}

%%%%%% RESUMO %%%%%%
\begin{abstract}
Resumo de 200-300 palavras.
\end{abstract}

%%%%%% Agradecimentos %%%%%%
% Segundo glisc deveria aparecer após conclusão...
\renewcommand{\abstractname}{Agradecimentos}
\begin{abstract}
Eventuais agradecimentos.
\end{abstract}


\tableofcontents
% \listoftables     % descomentar se necessário
% \listoffigures    % descomentar se necessário


%%%%%%%%%%%%%%%%%%%%%%%%%%%%%%%
\clearpage
\pagenumbering{arabic}

%%%%%%%%%%%%%%%%%%%%%%%%%%%%%%%%
\chapter{Introdução}
\label{chap.introducao}

Desde os primórdios que o Homem procura ter controlo da sua realidade moldando-a e modificando-a de modo a que as suas necessidades sejam supridas. Pode-se tomar como exemplo o controlo do fogo: quando o Homem primitivo descobriu como gerar artificialmente e controlar o fogo teve a sua vida facilitada e abriu um leque de novas possibilidades que originaram uma grande revolução a todos os níveis.

Passados alguns milhares de anos, o ser humano continua a tentar ter ainda mais controlo sobre a realidade de modo a que o impossível se torne possível. Como a \ac{ra} extende virtualmente aquilo que existe no mundo real, existe uma forte probabilidade de que, tal como o fogo, a \ac{ra} venha a revolucionar a forma como se vive e dar azo ao surgimento de novas possibilidades.

Apesar de ser uma tecnologia relativamente recente, a \ac{ra}te tido uma considerável evolução e, por isso, promete ser o futuro da tecnologia e integrar-se cada vez mais no dia a dia do cidadão comum. Apesar de não estar implementada em grande escala, esta tecnologia já tem aplicações vastas a nível empresarial. Áreas como a medicina, o entretenhimento, o design, a educação e a arquitetura poderão beneficiar dos novos recursos e funcionalidades criados por esta tecnologia.



Este documento está dividido em quatro capítulos.
Depois desta introdução,
no \autoref{chap.metodologia} é apresentada a metodologia seguida,
no \autoref{chap.resultados} são apresentados os resultados obtidos,
sendo estes discutidos no \autoref{chap.analise}.
Finalmente, no \autoref{chap.conclusao} são apresentadas
as conclusões do trabalho.

\chapter{Metodologia}
\label{chap.metodologia}

Neste relatório utilizou-se uma metodologia baseada, maioritariamente, na pesquisa exploratória. Esta pesquisa qualitativa vai nos permitir entender melhor e aprofundar o nosso tema para que o consigamos abordar da maneira mais clara e completa. Além disso, também teremos bases na leituar de algumas revistas bem conceituadas na área da tecnologia e ciencia, tal como o estudo de algumas investigações na área, para que nos permitam entender melhor o tema do nosso trabalho.


Neste esqueleto de relatório aproveitamos este capítulo para exemplificar
como se usam alguns elementos de {\LaTeX}.

\section{Exemplos}

\subsection{Utilização de acrónimos}
Esta é a primeira invocação do acrónimo \ac{ua}.
E esta é a segunda: \ac{ua}.

\subsection{Referências bibliográficas}
Informação relativa à estrutura formal de um relatório pode ser obtida
na página do \ac{glisc}\cite{glisc}.

\chapter{Realidade Aumentada- definição}
\label{chap.resultados}
Realidade Aumentada \ac{ra} ou \ac{ar} consiste na integração de elementos ou informações virtuais na visualização do mundo real através de uma câmera, com o uso de sensores de movimento como o giroscópio e o acelerômetro. O uso mais utilizado, e mais conhecido da realidade aumentada é o entretenimento, através dos filtros para fotos em aplicativos móveis de redes sociais, através de jogos como o Pokémon GO. A realidade aumentada é também utilizada de muitas formas nas áreas do ensino, design de produtos, ações de marketing, suporte em plantas industriais, entre outros. O uso de vídeos transmitidos ao vivo digitalmente processados e "ampliados" pela adição de gráficos criados pelo computador também podem ser considerados como um tipo de realidade aumentada. Um usuário da \ac{ra} pode utilizar uns óculos, ou câmeras acopladas a um dispositivo computacional, e através destes, poderá ver o mundo real bem como imagens geradas por computador projetadas no mundo. 
A \ac{ar} baseia-se numa experiência interativa entre um mundo real, onde objetos que pertencem ao mundo real podem ser alterados por informação perceptiva criada por computadores, podendo ser visual, auditiva, sensorial e olfativo. Pode ser construtiva (que acrescenta ao ambiente natural) ou destrutiva (que mascarpõe uma mascara sobreposta ao ambiente natural). A realidade aumentada altera o mundo real do usuário, enquanto a \ac{rv} substitui completamente o mundo real do expectador. A Realidade aumentada é relacionada a dois termos muito usados no meio tecnologico: a Realidade mista, e a Realidade mediada por computadores. 

-Realidade Mista - a realidade mista é a tecnologia que une as características da realidade virtual com a realidade aumentada. Permite inserir objetos virtuais num mundo real e permite a interação do usuário com os mesmos, produzindo um novo ambiente ao qual os itens físicos e virtuais coexistem e interagem em tempo real. Um exemplo da realidade mista é o \textit{head-up display} que encontramos nos carros mais modernos.

-Realidade Mediada - a realidade mediada consiste na capacidade de adicionar ou subtrair informação da precepção da realidade através da Utilização de um \textit{wearable computer} ou mesmo de um \textit{smartphone}, basta um dispositivo que permita criar um filtro visual entre o mundo real e aquilo que o utilizador capta, criando um cenário novo ao utiliador.

Voltando a falar da \ac{ra}, é ela que permite trazer componentes do mundo digital para dentro da percepção da pessoa do mundo real, e não o faz apenas dispondo as informações visualmente, mas também através da integração de sensações imersivas que são interpretadas como sendo algo pertencente a um ambiente.

\chapter{Análise}
\label{chap.analise}
Analisa os resultados.

\chapter{Conclusões}
\label{chap.conclusao}
Apresenta conclusões.

\chapter*{Contribuições dos autores}
Resumir aqui o que cada autor fez no trabalho.
Usar abreviaturas para identificar os autores,
por exemplo AS para António Silva.
No fim indicar a percentagem de contribuição de cada autor.

%%%%%%%%%%%%%%%%%%%%%%%%%%%%%%%%%
\chapter*{Acrónimos}
\begin{acronym}
\acro{ua}[UA]{Universidade de Aveiro}
\acro{leci}[LECI]{Licenciatura em Engenharia de Computadores e Informática}
\acro{ra}[RA]{Realidade Aumentada}
\acro{ar}[AR]{Augmented Reality}
\acro{rv}[RV]{Realidade Virtual}
\end{acronym}


%%%%%%%%%%%%%%%%%%%%%%%%%%%%%%%%%
\printbibliography

\end{document}
